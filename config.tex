\department{Wydział Matematyki, Fizyki i Informatyki}
\instytut{Instytut Informatyki}
\kierunek{Informatyka}
\specjalnosc{SPECJALNOŚĆ}
\title{Projekt i implementacja algorytmu symulowanego wyżarzania dla problemu
harmonogramowania zabiegów rehabilitacyjnych}
\engtitle{Design and implementation of a simulated annealing algorithm for
scheduling rehabilitation appointments}
\album{119838}
\author{Igor Motyka}
\promotor{dr inż. Sławomira Stemplewskiego}
\date{2021}

\wstep{
Optymalizacja procesów w placówkach medycznych jest kluczowa, jeśli chcemy
zredukować koszty operacyjne i tym samym realnie przyczynić się do spadku cen
usług świadczonych przez daną jednostkę medyczną. Komputerowe harmonogramowanie terminów zabiegów czy dyżurów
pielęgniarek powoduje redukcję personelu do tej pory odpowiedzialnego za manualne wykonywanie tych czynności.

Rozwiązanie problemu wyznaczania terminów rehabilitacji polega na znalezieniu takich bloków
czasowych w ciągu dnia, w których mogą zostać umieszczone terminy wykonania zabiegów. Każdy
pacjent ma zleconą przez lekarza listę koniecznych do wykonania zabiegów (np. 2x bieganie, 3x masaż).

Dane zebrane przez placówkę rehabilitacyjną pokazują, że średnia ilość wyznaczanych zabiegów dla jednego pacjenta to 80. Zabiegi mogą zostać wykonane tylko w pewnej sali zabiegowej, która znajduje się w danym bloku czasowych i ma swoją ograniczoną pojemność.

Pewne zabiegi wymagają konkretnego odstępu czasowego między sobą np. zabieg
\emph{laseroterapii} nie może być wykonywany w przeciągu 1h od momentu zakończenia
wykonania zabiegu \emph{krioterapii}. Taka reguła nie koniecznie może występować w obie
strony.

Terminy preferowane przez pacjenta, oraz godziny otwarcia danej placówki też
muszą być uwzględnione w procesie wyznaczania terminów odbycia zabiegów - tak samo
jak święta czy sytuacje nagłe takie jak zepsucie sprzętu niezbędnego do
przeprowadzenia danej procedury.

Dostępność terminów jest bardzo płynna i często zdarzają się sytuacje w których
dany zabieg może być wyznaczony na drugi dzień od momentu otrzymania zgłoszenia, a już kolejny zabieg na liście
zabiegów zleconych przez lekarza za 4 miesiące. Dlatego też przedział czasowy brany pod uwagę musi być odpowiednio duży (od 6
miesięcy do nawet roku).
\vskip 2em \parindent 0em
Jak widać po tym, krótkim i niepełnym opisie problem nie jest trywialny. Duża przestrzeń potencjalnych rozwiązań, typów zabiegów oraz ograniczeń - o których więcej w rozdziale \ref{constraints} - sprawia, że problem jest NP trudny.
}

\streszczeniepracy{Streszczenie pracy}
\slowakluczowe{Metaheurystyka, symulowane wyżarzanie, problem
harmonogramowania, optymalizacja procesów, hiperheurystyka}

\thesisabstract{Abstract}
\thesiskeywords{Metaheuristics, simulated annealing, tabu search, optimization
problem, scheduling problem, process automation}
