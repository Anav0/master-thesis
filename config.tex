\department{Wydział Matematyki, Fizyki i Informatyki}

\instytut{Instytut
Informatyki}

\kierunek{Informatyka}

\specjalnosc{SPECJALNOŚĆ}

\title{Projekt i
implementacja algorytmu symulowanego wyżarzania dla problemu harmonogramowania
zabiegów rehabilitacyjnych}

\engtitle{Design and implementation of a simulated
annealing algorithm for scheduling rehabilitation appointments}

\album{119838}

\author{Igor Motyka}

\promotor{dr inż. Sławomira Stemplewskiego}

\date{2021}

\wstep{ Optymalizacja procesów harmonogramowania w placówkach medycznych, jest
	ważnym czynnikiem w próbie redukcji kosztów operacyjnych danej
	jednostki. Komputerowe harmonogramowanie terminów zabiegów czy dyżurów
	pielęgniarek powoduje, redukcję personelu do tej pory odpowiedzialnego,
	za manualne harmonogramowanie tych procesów. Zmniejszenie ilości
	personelu, przy jednoczesnym zwiększeniu jakości uzyskiwanych
	harmonogramów, owocuje mniejszymi kosztami jakie ponosi placówka, a tym
	samym jakie musi pokryć klient.

W niniejszej pracy przedstawiona zostanie propozycja implementacji rozwiązania
problemu harmonogramowania zabiegów rehabilitacyjnych, który występuje w wielu
placówkach medycznych, współpracujących z Opolską firmą Medinet. Medinet
\footnote{Strona internetowa firmy - http://www.medinet.opole.pl/} jest to
Opolska firma dostarczająca oprogramowanie, umożliwiające kompleksową obsługę
wielu procesów niezbędnych w płynnym działaniu, każdej placówki służby zdrowia

Harmonogramowanie dat wykonania zabiegów rehabilitacyjnych, jest pod wieloma
względami unikatowe, na tle innych problemów harmonogramowania, takich jak
układanie planu lekcji czy planowanie zmian w jakich mają pracować pielęgniarki.
Zabiegi rehabilitacyjne muszą się odbywać w cyklach, tak aby pacjent faktycznie
skorzystał z leczniczych efektów danego zabiegu; Zabiegi nie mogą być wykonywane
w dowolnej kolejności np. zabieg laseroterapii nie może się odbyć, jeśli pacjent
nie poczeka minimum 20 min, w przypadku gdy poprzednim zabiegiem był zabieg
krioterapii; Dany typ zabiegu, nie może się odbywać częściej, niż raz dziennie;
Pacjent może mieć bardzo konkretne preferencje, co do kolejności i czasu
wykonywania zabiegów. Jeden pacjent może preferować wykonanie zabiegu już w
następnym tygodniu od daty otrzymania skierowania, a inny może preferować
następny miesiąc czy kwartał.}

\streszczeniepracy{Streszczenie pracy}

\slowakluczowe{Metaheurystyka, symulowane
wyżarzanie, problem harmonogramowania, optymalizacja procesów}

\thesisabstract{Abstract}

\thesiskeywords{Metaheuristics, simulated annealing,
optimization problem, scheduling problem, process automation}
