\chapter{Przegląd literatury}
W tym rozdziale zostały opisane wnioski, wyciągnięte z przeglądu dotychczasowych
prac badawczych traktujących o problemach harmonogramowania.
\section{Możliwe rozwiązania}
Do rozwiązania problemu harmonogramowania można użyć szeregu technik, algorytmów czy też sposobów. Badacze z powodzeniem zastosowali chociażby algorytm genetyczny, programowania liniowe oraz symulowane wyżarzanie\cite{marzec} do rozwiązania tego typu problemów.
\subsection{Algorytm genetyczny}
\subsection{Programowanie liniowe}
\subsection{Symulowane wyżarzanie}