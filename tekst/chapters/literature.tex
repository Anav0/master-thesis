\chapter{Przegląd literatury}
W tym rozdziale zostały opisane najpopularniejsze techniki stosowane przy
rozwiązywaniu problemów harmonogramowania.
\section{Możliwe rozwiązania}
Do rozwiązania problemu harmonogramowania można użyć szeregu technik, algorytmów
czy też sposobów. Badacze z powodzeniem zastosowali chociażby algorytm
genetyczny \cite{assi2018}, programowania matematyczne \cite{yakoob2007} czy symulowane wyżarzanie \cite{marzec2020}.

Szczegółową analizą stosowanych metod zajęli się S.A MirHassani i F. Habib
\cite{habib2013}. Metody te można podzielić na kilka grup:

\subsubsection{Metody sekwencyjne (z ang. \emph{sequential methods})}
Korzystają z algorytmów deterministycznych ściśle uzależnionych od konkretnego
problemu w celu uzyskania ułożenia wydarzeń w pożądanej kolejności. Np. od
najistotniejszych do najmniej istotnych. Następnie każdemu z
ułożonych wydarzeń przypisuje się slot czasowy (termin) tak aby nie powstał
konflikt. Im wcześniej na liści znajduje się dane wydarzenie, tym korzystniejszy
otrzyma slot czasowy.

\subsubsection{Metody klastrowe (z ang. \emph{claustering methods})} 
Reprezentują problem jako klastry spełniające
ograniczenia. Klastry składające się z wydarzeń są następnie przypisywane do
terminów, tak aby jak najlepiej spełnić stawiane preferencje. Grupowanie
wydarzeń w klastry ma miejsce na początku procesu wyznaczania i może skutkować
problemem w znalezieniu dobrego rozwiązania.

\subsubsection{Metody bazujące na ograniczeniach (z ang. \emph{constraint based methods})} 
Opisują problem jako zbiór wydarzeń, który
następnie przypisywane są zasoby tak aby spełnione zostały ograniczenia.

\subsubsection{Metaheurystyki (z ang. \emph{metaheuristics methods})}
Metaheurystkami nazywany algorytmy niedeterministyczne, które są w stanie ocenić
w czasie wielomianowym, jakość danego rozwiązania. Do grupy algorytmów
metaheurystycznych zaliczamy: algorytm genetyczny, symulowane wyżarzanie,
algorytm mrówkowy, tabu search etc.
\section{Najpopularniejsze podejścia do rozwiązania problemu harmonogramowania}
MirHassani i Habib'a zbadali częstotliwość wykorzystywania różnych
algorytmów, oto najpopularniejsze z nich:
\subsection{Algorytm genetyczny (z ang. \emph{genetic algorithm})}
Podejście genetyczne \cite{komosinski2021} polega na reprezentacji rozwiązania za pomocą sekwencji
bitów o stałej długości. Taka sekwencja jest nazywana genotypem. Zbiór genotypów nazywamy
populacją. Populacja jest poddawana pewnym procesom mającym naśladować procesy
zachodzące podczas doboru naturalnego. Algorytm zatem definiuje pewne parametry:
\begin{itemize}
	\item Wielkość populacji \emph{n}.
	\item Prawdopodobieństwo krzyżowania \emph{cross}.
	\item Prawdopodobieństwo mutacji \emph{mut}.
\end{itemize}

\begin{algorithm}
\caption{Algorytm genetyczny}\label{alg-genetic}
\begin{algorithmic}[1]
\Procedure{Genetic}{$cross,mut,n,stopCriteria$}
\State $t\gets 0$\Comment{Iteracja}
\State $Init(P(t),n)$\Comment{Stwórz populację}
\State $Eval(P(t))$\Comment{Oceń populację}
\While{$stopCriteria()\not=true$}
\State $t\gets t+1$
\State $P(t)\gets P(t-1)$\Comment{Stwórz nową populację z poprzedniej}
\State $Change(P(t),mut,cross)$\Comment{Zmień populację}
\State $Eval(P(t))$\Comment{Oceń nową populację}
\EndWhile\label{euclidendwhile}
\State $\textbf{return}\ Best(P(t))$
\EndProcedure
\end{algorithmic}
\end{algorithm}

Istnieją dwa podejścia co do zmiany populacji. Podejście inkrementacyjne i
podejście stanu stałego. Podejście inkrementacyjne za każdym razem zmienia tylko
część istniejącej populacji, podczas gdy podejście stanu stałego za każdym razem
tworzy nową populację w oparciu o tę poprzednią. Nowa populacja składa się z
potomków osobników z poprzedniej populacji.

\subsection{Przeszukiwanie z tabu (z ang. \emph{tabu search})}
Przeszukiwanie tabu definiuje pewną listą tabu. Lista tabu składa się z par
wartości. Pierwsza wartość definiuje nam rozwiązanie a druga czas, jaki musi
upłynąć zanim algorytm znów będzie mógł "wrócić" do danego rozwiązania.

Zastosowanie tej techniki pozwala wymusić na algorytmie sprawdzenie większej
ilości unikatowych rozwiązań, bez cyklicznego wracania do tych już napotkanych
przez algorytm.

\subsection{Kolorowanie grafu (z ang. \emph{graph coloring})}
Problem harmonogramowania można przekształcić w problem kolorowania grafu \cite{haan2006}.
Wierzchołki grafu $v \in V$ reprezentują spotkania/wydarzenia, krawędzie
$e \in E$ reprezentują konflikty między nimi a kolory wierzchołków reprezentują sloty czasowe
(terminy). 

Niestety rozwiązania uzyskiwane po rozwiązaniu problemu kolorowania grafu nie są
dostatecznie zadowalające \cite{haan2006}, dlatego stosuje się specjalne fazy, 
które mają za zadanie zwiększyć jakość ostatecznego rozwiązania. W pierwszych
fazach uzyskuje się rozwiązanie dopuszczalne a w kolejnych wykorzystuje się
przeszukiwanie z tabu lub inną metaheurystkę w celu ulepszenia dopuszczalnego
rozwiązania.

\subsection{Programowanie matematyczne (z ang. \emph{Integer programming})}
Reprezentuje problem jako zbiór liczb całkowitych. Z reguły do reprezentacji
problemu stosuje się liczby całkowite oraz liczby binarne
\cite{maaroufi2016mixed}. Jest to forma zbliżona do tej z której korzysta algorytm genetyczny.

Rozwiązaniem problemu jest takie dobranie wartości liczbowych reprezentujących
problem, aby jak najlepiej spełnić ograniczenia oraz preferencje.

Ta technika jest bardzo popularna, ponieważ całkowitoliczbowa reprezentacja problemu daje
większą kontrolę nad operacjami przeprowadzanymi na rozwiązaniu niż ma to miejsce w algorytmie
genetycznym. Mutacja czy krzyżowanie zmienia bity bez wiedzy o tym, jaki aspekt
rozwiązania jest przez nie kontrolowany. Niestety czasami nie można dokonać reprezentacji problemu jako jeden wzór.

\section{Podsumowanie}
Jak widać możliwych do zastosowania algorytmów nie brakuje. Algorytm genetyczny
mimo iż najpopularniejszy, nie pozwala na dokładną kontrolę tego, co jest
zmieniane w naszym rozwiązaniu. Przekształcenie problemu w problem kolorowania
grafu wymaga dodatkowych kroków, które by poprawiły uzyskane rozwiązanie.
Programowanie całkowitoliczbowe eliminuje część problemów napotkanych w
algorytmie genetycznym, jednak ma ono problemy z dużymi instancjami problemu
harmonogramowania.

Z powyższych powodów, algorytmem wybranym do rozwiązania problemu
harmonogramowania zabiegów rehabilitacyjnych jest symulowane wyżarzanie,
szczegółowo opisane w rozdziale \ref{annealing}.
