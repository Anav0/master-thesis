\chapter{Badania efektywności zaproponowanego rozwiązania}
\section{Metodologia}
Aby sprawdzić efektywność zaproponowanego algorytmu, należy go uruchomić dla różnych instancji problemu
z konkretnymi parametrami oraz dla różnych przestrzeni rozwiązań. Każdą instancję trzeba rozwiązać $n=10$ razy,
tak aby uzyskać wiarygodny obraz jakości algorytmu nie
zniekształcony normalnymi wahaniami niedeterminizmu. Podczas każdego uruchomienia symulowanego wyżarzania,
zapisywana jest wartość najlepszego znalezionego rozwiązania w danej iteracji. Jeśli przez pierwsze 100 iteracji,
średnia wartość najlepszego znalezionego rozwiązanie ma wartość
$0.25$ to taka informacja zostanie zaznaczona na wykresie.

Uzyskane w ten sposób dane o zmianie wartości w czasie, należy
znormalizować w obrębie konkretnej instancji do zakresu od 0 do 1. Najlepsza wartość
pośród $n$ uruchomień danej instancji jest naszym 0 a najgorsza wartość
(początkowa) naszą 1.

\subsection{Instancje badawcze}
Ponieważ ilość zabiegów w zleceniu waha się (rys. \ref{medinet-treatments-stats}) między 10 a 70
zatem rozsądne jest skomponowanie listy instancji badawczych w oparciu o kilka
typowych zleceń danej wielkości. Dla każdej wielkości zleceń, wybrano jako
instancje badawcze dwa zlecenia różniące się od siebie typami zleconych zabiegów.
Szczegółowo zostały przedstawione instancje wielkości 10 i 30.
\newpage
\subsection{Badana przestrzeń rozwiązań}
Przestrzeń rozwiązań różni się w zależności od wielkości zlecenia oraz dat
ograniczających proces wyznaczania. Sprawdzenie każdej możliwej kombinacji daty
początkowej i końcowej jest niemożliwe. Przewidzenie jak w przyszłości będzie
wyglądał krajobraz funkcji oceny jest również niemożliwe. Warto zatem sprawdzić
takie zakresy, które będą trudne dla algorytmu. Trudne zakresy to takie, które
zawierają w sobie dużą ilość już wyznaczonych zabiegów. Sprawdzenie trudnych
zakresów pozwoli nam ocenić jak w przyszłości będzie sobie radził program z
jeszcze nieznanymi przestrzeniami rozwiązań.

\begin{table}[H]
    \centering
    \begin{tabular}[width=\textwidth]{ | l | l | l |}
        \hline
        \bfseries Nazwa skrótowa & \bfseries Data początkowa & \bfseries Data końcowa \\
        \hline
        Zakres 1 & 2018-01-01 08:00 & 2018-08-01 18:00 \\
        \hline
        Zakres 2 & 2018-03-01 08:00 & 2018-10-01 18:00 \\
        \hline
        Zakres 3 & 2018-06-01 08:00 & 2018-12-01 18:00 \\
        \hline
        Zakres 4 & 2018-09-01 08:00 & 2019-03-01 18:00 \\
        \hline
    \end{tabular}
    \caption{Zakresy dat wykorzystane w badaniu}
\end{table}

\subsection{Parametry wykorzystane w badaniu}
Parametry SA zostały dobrane pod konkretne rozmiary badanych instancji tak, aby efekt
działania algorytmu był jak najlepszy.
\begin{table}[H]
\centering
\begin{tabular}[\textwidth]{ | l | l | l | l | l | l | l | l | l | }
\hline
\bfseries Rozmiar & \bfseries Temperatura & \bfseries Tempo wychładzania
	     & \bfseries Maks. ilość iteracji & \bfseries Maks. bez polepszenia\\
\hline
10 & 500.00 & 0.9550 & 25,000.00 & 5,000.00 \\
\hline
20 & 500.00 & 0.9550 & 50,000.00 & 10,000.00 \\
\hline
30 & 500.00 & 0.9550 & 50,000.00 & 10,000.00 \\
\hline
40 & 500.00 & 0.9550 & 50,000.00 & 10,000.00 \\
\hline
50 & 500.00 & 0.9950 & 50,000.00 & 10,000.00 \\
\hline
60 & 500.00 & 0.9950 & 50,000.00 & 10,000.00 \\
\hline
70 & 500.00 & 0.9950 & 50,000.00 & 10,000.00 \\
\hline
\end{tabular}
\caption{Lista parametrów wykorzystanych w badaniu}
\end{table}

\newpage
\section{Szczegółowe wyniki badań dla wybranych instancji}
\subsection{Badanie skuteczności rozwiązywania instancji 10 elementowych}
\begin{table}[H]
\begin{tabular}{ l l }
	Średni czas działania: & 3.5598 (s) \\
Średnia wartość: & 0.3828 \\
\end{tabular}
\end{table}
\begin{table}[H]
\centering
\begin{tabularx}{1\textwidth}{ | l | X | }
\hline
\bfseries Parametr & \bfseries Wartość \\
\hline
Temperatura & 500.00 \\
\hline
Tempo wychładzania & 0.9950 \\
\hline
Maksymalna ilość iteracji & 25,000.00 \\
\hline
Zatrzymaj, jeśli nie uzyska polepszenia przez x iteracji & 5,000.00 \\
\hline
\end{tabularx}
\caption{Parametry dla zleceń zawierających 10 zabiegów}
\end{table}

\begin{table}[H]
\centering
\begin{tabularx}{1\textwidth}{ | X | l | }
\hline
\bfseries Nazwa zabiegu & \bfseries Ilość do wykonania \\
\hline
MAGNETRONIK & 10 \\
\hline
\end{tabularx}
\caption{Szczegóły badanego zlecenia 10 elementowego}
\end{table}
\begin{figure}[H]
\centering
\begin{table}[H]
\centering
\begin{tabularx}{1\textwidth}{ c c }
\includegraphics[width=0.5\textwidth]{gfx/test-plots/62713_temp_all} & \includegraphics[width=0.5\textwidth]{gfx/test-plots/62713_temp_part} \\
\end{tabularx}
\end{table}
\caption{Spadek temperatury dla instancji 10 elementowych.}
\end{figure}
\begin{figure}[H]
\centering
\includegraphics[width=0.7\textwidth]{gfx/test-plots/62713_best_all}
\caption{Znormalizowana wartość najlepszego rozwiązania w danej iteracji (10).}
\label{temp-10}
\end{figure}
\subsubsection{Uwagi} 
Jak widać po rysunku \ref{temp-10}, algorytm dla zlecenia
zawierającego tylko zabiegi tego samego typu, nie znalazł lepszego rozwiązania
niż rozwiązanie początkowe, ponieważ rozwiązanie początkowe już wstawiło każde zabieg na osobny dzień, tak blisko początku dnia jak to tylko możliwe, co dobrze spełnia zdefiniowane preferencje.

\newpage
\subsection{Szczegóły badania na zleceniu 30 elementowym}
\begin{table}[H]
\begin{tabular}{ l l }
	Średni czas działania: & 5.0356 (s) \\
Średnia wartość: & 0.7326 \\
\end{tabular}
\end{table}

\begin{table}[H]
\centering
\begin{tabularx}{1\textwidth}{ | l | X | }
\hline
\bfseries Parametr & \bfseries Wartość \\
\hline
Temperatura & 500.00 \\
\hline
Tempo wychładzania & 0.9950 \\
\hline
Maksymalna ilość iteracji & 50,000.00 \\
\hline
Zatrzymaj, jeśli nie uzyska polepszenia przez x iteracji & 10,000.00 \\
\hline
\end{tabularx}
\caption{Parametry dla zleceń zawierających 30 zabiegów}
\end{table}

\begin{table}[H]
\centering
\begin{tabularx}{1\textwidth}{ | X | l | }
\hline
\bfseries Nazwa zabiegu & \bfseries Ilość do wykonania \\
\hline
Krioterapia AZOT & 10 \\
\hline
MAGNETRONIK & 10 \\
\hline
Laser punktowy & 10 \\
\hline
\end{tabularx}
\caption{Szczegóły badanego zlecenia 30 elementowego}
\end{table}

\begin{figure}[H]
\centering
\begin{table}[H]
\centering
\begin{tabularx}{1\textwidth}{ c c }
\includegraphics[width=0.5\textwidth]{gfx/test-plots/78636_temp_all} & \includegraphics[width=0.5\textwidth]{gfx/test-plots/78636_temp_part} \\

\end{tabularx}
\end{table}
\caption{Spadek temperatury dla instancji 30 elementowych.}
\end{figure}
\begin{figure}[H]
\centering
\includegraphics[width=0.7\textwidth]{gfx/test-plots/78636_best_all}
\caption{Znormalizowana wartość najlepszego rozwiązania w danej iteracji (30).}
\end{figure}
\begin{figure}[H]
\centering
\includegraphics[width=14cm]{gfx/test-plots/78636_deviation}
\caption{Odchylenie między wartościami najlepszego rozwiązania w danej iteracji (30).}
\end{figure}

%\begin{figure}[H]
%\centering
%\includegraphics[width=1\textwidth]{gfx/test-plots/all_in_one}
%\caption{Wszystkie znormalizowane wartość najlepszego rozwiązania w danej iteracji}
%\end{figure}

\subsection{Podsumowanie wyników}
\begin{table}[H]
\centering
\begin{tabularx}{1\textwidth}{ | X | X | l | l | l | l | l | l | X | }
\hline
\bfseries Id & \bfseries Rozmiar & $\mathbf{v_1}$ & $\mathbf{v_2}$ & $\mathbf{v_3}$ & $\mathbf{v_4}$ & $\mathbf{v_5}$ & \bfseries Suma & \bfseries Średni czas (s) \\
\hline
62713 & 10 & 0.3113 & 0.0000 & 0.0000 & 0.0115 & 0.0600 & 0.3828 & 3.5598 \\
\hline
74543 & 10 & 0.3113 & 0.0000 & 0.0000 & 0.0120 & 0.0600 & 0.3833 & 3.5550 \\
\hline
78587 & 20 & 0.4415 & 0.0000 & 0.2450 & 0.0786 & 0.0600 & 0.8252 & 4.2552 \\
\hline
83933 & 20 & 0.4275 & 0.0000 & 0.2554 & 0.0778 & 0.0600 & 0.8207 & 4.1729 \\
\hline
78636 & 30 & 0.4207 & 0.0370 & 0.1397 & 0.0735 & 0.0617 & 0.7326 & 5.0356 \\
\hline
78717 & 30 & 0.4492 & 0.1301 & 0.1393 & 0.0712 & 0.0660 & 0.8559 & 4.9717 \\
\hline
150165 & 40 & 0.3803 & 0.8788 & 0.0478 & 0.0225 & 0.1820 & 1.5115 & 4.7056 \\
\hline
155280 & 40 & 0.4335 & 0.3032 & 0.1417 & 0.0721 & 0.0673 & 1.0177 & 6.0482 \\
\hline
178354 & 50 & 0.4000 & 0.9004 & 0.0468 & 0.0198 & 0.2201 & 1.5870 & 5.3316 \\
\hline
180837 & 50 & 0.3900 & 0.2480 & 0.1138 & 0.0733 & 0.0990 & 0.9241 & 6.0106 \\
\hline
80962 & 60 & 0.4150 & 0.4923 & 0.1114 & 0.0774 & 0.0769 & 1.1731 & 7.8611 \\
\hline
84035 & 60 & 0.3865 & 0.4723 & 0.1203 & 0.0799 & 0.0748 & 1.1338 & 7.2471 \\
\hline
246510 & 70 & 0.4152 & 0.6144 & 0.1249 & 0.0772 & 0.0957 & 1.3274 & 8.1094 \\
\hline
274804 & 70 & 0.4204 & 0.8417 & 0.0471 & 0.0399 & 0.2246 & 1.5738 & 6.6644 \\
\hline
\end{tabularx}
\caption{Podsumowanie wyników badań}
\end{table}
Algorytm najlepiej spełnił preferencję $v_5$ i $v_4$. Największą trudność
algorytmowi sprawiło spełnienie preferencji $v_1$, której wartość jest tym mniejsza im
mniejsza jest odległość od daty początkowej. Nietrudno zauważyć, że algorytm
większy nacisk kładł na znalezienie rozwiązania, które nie wymaga aby
pacjent długo przebywał w klinice.

Zlecenie o tym samym rozmiarze nie zawsze są równie skutecznie rozwiązywane. Algorytm bardzo różnie ocenił
zlecenia 50 elementowe, szczególnie duża różnica jest widoczna w warunku $v_2$. Różnice w wartościowaniu wynikają, z różnej
kompozycji tych zleceń.

Średnia wartość nawet dużych instancji jest niewielka bo wynosi 1.5738, co
oznacza, że algorytm dobrze poradził sobie z rozwiązywaniem badanych instancji.
