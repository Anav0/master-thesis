\chapter{Badania efektywności zaproponowanego rozwiązania}
Rozdział ten opisuje badania mające na celu sprawdzenie efektywności
zaproponowanego rozwiązania.

\section{Metodologia}
Aby sprawdzić efektywność zaproponowanego algorytmu, należy go uruchomić dla różnych instancji problemu
z konkretnymi parametrami oraz dla różnych przestrzeni rozwiązań. Każdą instancję należy rozwiązać $n$ razy,
tak aby uzyskać wiarygodny obraz jakości algorytmu nie
zniekształcony normalnymi wahaniami niedeterminizmu. Podczas każdego uruchomienia symulowanego wyżarzania,
zapisywana jest wartość najlepszego znalezionego rozwiązania w danej iteracji. Jeśli przez pierwsze 100 iteracji,
średnia wartość najlepszego znalezione rozwiązanie ma wartość
$0.25$ to taka informacja zostanie wyświetlona na wykresie.

Uzyskane w ten sposób dane o zmianie wartości w czasie, należy
znormalizować w obrębie konkretnej instancji do zakresu od 0 do 1. Najlepsza wartość
pośród $n$ uruchomień danej instancji jest naszym 0 a najgorsza wartość
(początkowa) naszą 1.

\subsection{Instancje testowe}
Ponieważ ilość zabiegów w zleceniu waha się (rys. \ref{medinet-treatments-stats}) między 10 a 70
zatem rozsądne jest skomponowanie listy instancji testowych w oparciu o kilka
typowych zleceń danej wielkości. Dla każdej wielkości zleceń, wybrano jako
instancje testowe dwa zlecenia różniące się od siebie typami zleconych zabiegów.
Szczegółowo zostały przedstawione instancje wielkości 10, 30 i 50.
\newpage
\subsection{Testowa przestrzeń rozwiązań}
Przestrzeń rozwiązań różni się w zależności od wielkości zlecenia
oraz dat ograniczających proces wyznaczania. Sprawdzenie każdej
możliwej kombinacji daty początkowej i końcowej jest niemożliwe. Przewidzenie
jak w przyszłości będzie wyglądał krajobraz funkcji oceny w ciągle zmieniającym
się kalendarzu jest również niemożliwe. Warto zatem sprawdzić
jakość algorytmu wybierając taki zakres dat, który dobrze reprezentuje faktyczny
stan kalendarza placówki.

\begin{table}[H]
    \centering
    \begin{tabular}[width=\textwidth]{ | l | l | l |}
        \hline
        \bfseries Nazwa skrótowa & \bfseries Data początkowa & \bfseries Data końcowa \\
        \hline
        Zakres 1 & 2018-01-01 08:00 & 2018-08-01 18:00 \\
        \hline
        Zakres 2 & 2018-03-01 08:00 & 2018-10-01 18:00 \\
        \hline
        Zakres 3 & 2018-06-01 08:00 & 2018-12-01 18:00 \\
        \hline
        Zakres 4 & 2018-09-01 08:00 & 2019-03-01 18:00 \\
        \hline
    \end{tabular}
    \caption{Zakresy dat wykorzystane w badaniu}
\end{table}

\subsection{Parametry wykorzystane w badaniu}
Parametry SA zostały dobrane pod konkretne rozmiary instancji testowych, tak aby efekt
działania algorytmu był jak najlepszy dla danej instancji.
\begin{table}[H]
\centering
\begin{tabular}[\textwidth]{ | l | l | l | l | l | l | l | l | l | }
\hline
\bfseries Rozmiar & \bfseries Temperatura & \bfseries Tempo wychładzania
	     & \bfseries Maks. ilość iteracji & \bfseries Maks. bez polepszenia\\
\hline
10 & 500.00 & 0.9550 & 25,000.00 & 5,000.00 \\
\hline
20 & 500.00 & 0.9550 & 50,000.00 & 10,000.00 \\
\hline
30 & 500.00 & 0.9550 & 50,000.00 & 10,000.00 \\
\hline
40 & 500.00 & 0.9550 & 50,000.00 & 10,000.00 \\
\hline
50 & 500.00 & 0.9950 & 50,000.00 & 10,000.00 \\
\hline
60 & 500.00 & 0.9950 & 50,000.00 & 10,000.00 \\
\hline
70 & 500.00 & 0.9950 & 50,000.00 & 10,000.00 \\
\hline
\end{tabular}
\caption{Lista parametrów wykorzystanych w badaniu}
\end{table}

\newpage
\section{Szczegółowe wyniki badań dla wybranych instancji}
\subsection{Badanie skuteczności rozwiązywania instancji 10 elementowych}
\begin{table}[H]
\begin{tabular}{ l l }
	Średni czas działania: & 3.5598 (s) \\
Średnia wartość: & 0.3828 \\
\end{tabular}
\end{table}
\begin{table}[H]
\centering
\begin{tabularx}{1\textwidth}{ | l | X | }
\hline
\bfseries Parametr & \bfseries Wartość \\
\hline
Temperatura & 500.00 \\
\hline
Tempo wychładzania & 0.9950 \\
\hline
Maksymalna ilość iteracji & 25,000.00 \\
\hline
Zatrzymaj, jeśli nie uzyska polepszenia przez x iteracji & 5,000.00 \\
\hline
\end{tabularx}
\caption{Parametry dla zleceń zawierających 10 zabiegów}
\end{table}

\begin{table}[H]
\centering
\begin{tabularx}{1\textwidth}{ | X | l | }
\hline
\bfseries Nazwa zabiegu & \bfseries Ilość do wykonania \\
\hline
MAGNETRONIK & 10 \\
\hline
\end{tabularx}
\caption{Szczegóły testowanego zlecenia 62713}
\end{table}
\begin{figure}[H]
\centering
\begin{table}[H]
\centering
\begin{tabularx}{1\textwidth}{ c c }
\includegraphics[width=0.5\textwidth]{gfx/test-plots/62713_temp_all} & \includegraphics[width=0.5\textwidth]{gfx/test-plots/62713_temp_part} \\
\end{tabularx}
\end{table}
\caption{Wykres spadku temperatury dla instancji 10 elementowych (Całkowity - lewy i zbliżony - prawy)}
\end{figure}
\begin{figure}[H]
\centering
\includegraphics[width=0.7\textwidth]{gfx/test-plots/62713_best_all}
\caption{Znormalizowana wartość najlepszego rozwiązania w danej iteracji dla instancji 10 elementowej. Średnia z 10 uruchomień.}
\label{temp-10}
\end{figure}
\subsubsection{Uwagi} 
Jak widać po rysunku \ref{temp-10}, algorytm dla zlecenia
zawierającego tylko zabiegi tego samego typu, nie znalazł lepszego rozwiązania
niż rozwiązanie początkowe, ponieważ rozwiązanie początkowe już wstawiło każde zabieg na osobny dzień.

\newpage
\subsection{Szczegóły badania zlecenia 30 elementowego}
\begin{table}[H]
\begin{tabular}{ l l }
	Średni czas działania: & 5.0356 (s) \\
Średnia wartość: & 0.7326 \\
\end{tabular}
\end{table}

\begin{table}[H]
\centering
\begin{tabularx}{1\textwidth}{ | l | X | }
\hline
\bfseries Parametr & \bfseries Wartość \\
\hline
Temperatura & 500.00 \\
\hline
Tempo wychładzania & 0.9950 \\
\hline
Maksymalna ilość iteracji & 50,000.00 \\
\hline
Zatrzymaj, jeśli nie uzyska polepszenia przez x iteracji & 10,000.00 \\
\hline
\end{tabularx}
\caption{Parametry dla zleceń zawierających 30 zabiegów}
\end{table}

\begin{table}[H]
\centering
\begin{tabularx}{1\textwidth}{ | X | l | }
\hline
\bfseries Nazwa zabiegu & \bfseries Ilość do wykonania \\
\hline
Krioterapia AZOT & 10 \\
\hline
MAGNETRONIK & 10 \\
\hline
Laser punktowy & 10 \\
\hline
\end{tabularx}
\caption{Szczegóły testowanego zlecenia 78636}
\end{table}

\begin{figure}[H]
\centering
\begin{table}[H]
\centering
\begin{tabularx}{1\textwidth}{ c c }
\includegraphics[width=0.5\textwidth]{gfx/test-plots/78636_temp_all} & \includegraphics[width=0.5\textwidth]{gfx/test-plots/78636_temp_part} \\

\end{tabularx}
\end{table}
\caption{Wykres spadku temperatury dla instancji 30 elementowych (Całkowity - lewy i zbliżony - prawy)}
\end{figure}
\begin{figure}[H]
\centering
\includegraphics[width=0.7\textwidth]{gfx/test-plots/78636_best_all}
\caption{Znormalizowana wartość najlepszego rozwiązania w danej iteracji dla instancji 30 elementowej. Średnia z 10 uruchomień.}
\end{figure}
\begin{figure}[H]
\centering
\includegraphics[width=14cm]{gfx/test-plots/78636_deviation}
\caption{Odchylenie między wartością najlepszego znalezionego rozwiązania w danej iteracji i danym zakresie dla instancji 30 elementowej}
\end{figure}

%\begin{figure}[H]
%\centering
%\includegraphics[width=1\textwidth]{gfx/test-plots/all_in_one}
%\caption{Wszystkie znormalizowane wartość najlepszego rozwiązania w danej iteracji}
%\end{figure}

\subsection{Podsumowanie wyników}
\begin{table}[H]
\centering
\begin{tabularx}{1\textwidth}{ | X | X | l | l | l | l | l | l | X | }
\hline
\bfseries Id & \bfseries Rozmiar & $\mathbf{v_1}$ & $\mathbf{v_2}$ & $\mathbf{v_3}$ & $\mathbf{v_4}$ & $\mathbf{v_5}$ & \bfseries Suma & \bfseries Średni czas (s) \\
\hline
62713 & 10 & 0.3113 & 0.0000 & 0.0000 & 0.0115 & 0.0600 & 0.3828 & 3.5598 \\
\hline
74543 & 10 & 0.3113 & 0.0000 & 0.0000 & 0.0120 & 0.0600 & 0.3833 & 3.5550 \\
\hline
78587 & 20 & 0.4415 & 0.0000 & 0.2450 & 0.0786 & 0.0600 & 0.8252 & 4.2552 \\
\hline
83933 & 20 & 0.4275 & 0.0000 & 0.2554 & 0.0778 & 0.0600 & 0.8207 & 4.1729 \\
\hline
78636 & 30 & 0.4207 & 0.0370 & 0.1397 & 0.0735 & 0.0617 & 0.7326 & 5.0356 \\
\hline
78717 & 30 & 0.4492 & 0.1301 & 0.1393 & 0.0712 & 0.0660 & 0.8559 & 4.9717 \\
\hline
150165 & 40 & 0.3803 & 0.8788 & 0.0478 & 0.0225 & 0.1820 & 1.5115 & 4.7056 \\
\hline
155280 & 40 & 0.4335 & 0.3032 & 0.1417 & 0.0721 & 0.0673 & 1.0177 & 6.0482 \\
\hline
80962 & 60 & 0.4150 & 0.4923 & 0.1114 & 0.0774 & 0.0769 & 1.1731 & 7.8611 \\
\hline
84035 & 60 & 0.3865 & 0.4723 & 0.1203 & 0.0799 & 0.0748 & 1.1338 & 7.2471 \\
\hline
246510 & 70 & 0.4152 & 0.6144 & 0.1249 & 0.0772 & 0.0957 & 1.3274 & 8.1094 \\
\hline
274804 & 70 & 0.4204 & 0.8417 & 0.0471 & 0.0399 & 0.2246 & 1.5738 & 6.6644 \\
\hline
178354 & 50 & 0.4000 & 0.9004 & 0.0468 & 0.0198 & 0.2201 & 1.5870 & 5.3316 \\
\hline
180837 & 50 & 0.3900 & 0.2480 & 0.1138 & 0.0733 & 0.0990 & 0.9241 & 6.0106 \\
\hline
\end{tabularx}
\caption{Wyniki badań dla wszystkich badanych instancji}
\end{table}
Jak widać na powyższej tabeli, czas działania algorytmu zwiększa się wraz z
rozmiarem instancji testowej. Dzieje się tak dlatego, że wartość określająca
maksymalną ilość kroków bez poprawy jest zwiększa dla większych instancji, przez
co algorytm wykonuje więcej iteracji.
