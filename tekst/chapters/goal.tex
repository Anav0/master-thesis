\chapter{Cel i zakres pracy} Celem niniejszej pracy jest rozwiązanie problemu
wyznaczania szeregu terminów rehabilitacji za pomocą algorytmu symulowanego
wyżarzania. Oprócz samego rozwiązania problemu, szczegółowo zostanie opisany
wpływ następujących czynników na szybkość działania algorytmu oraz na jakość
uzyskiwanych rozwiązań:

\begin{itemize} \item Temperatura początkowa \item Funkcja wychładzania \item
	Typ funkcji przejścia \item Wielkość instancji (ilość wizyt do
	wyznaczenia) \item Wielkość przestrzeni rozwiązań \item Ograniczenia
	twarde \item Ograniczenia miękkie \end{itemize} Po zbadaniu wpływu
	powyższych czynników na czas działania oraz jakość uzyskanych rozwiązań,
	różne połączenia parametrów - za wyjątkiem ograniczeń - również zostaną
	przebadane a wyniki szczegółowo opisane.

Szczególna uwaga zostanie poświęcona zbadaniu relacji między wielkością
instancji problemu a parametrami, dokładniej: czy optymalnie dobrane parametry
dla pewnej wielkości instancji dadzą równie dobre rezultaty dla instancji o
innej wielkości, oraz innej kompozycji (zawierającej inne typy wizyt).
