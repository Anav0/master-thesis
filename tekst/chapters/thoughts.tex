\chapter{Uwagi i przemyślenia}
W tym rozdziale opisane zostały ogólne uwagi, przemyślenia oraz trudności
powstałe podczas tworzenia programu.

\subsubsection{Parametryzacja SA}
Dostosowanie parametrów symulowanego wyżarzania oraz różnych szczegółów
implementacji do konkretnego problemu nie jest sprawą trywialną, tak było
również w tym przypadku. Tempo wychładzania, temperatura początkowa, rozwiązanie
początkowe - to od tych elementów zależy powodzenie algorytmu. Źle dobrane
parametry mogą w najlepszym razie znacząco spowolnić algorytm lub doprowadzić do
stanu w którym algorytm nie będzie zwracał dobrych rozwiązań.

\subsubsection{Wykorzystanie memoizacji do przyśpieszenia algorytmu}
Aby uzyskać odpowiednio niski czas działania programu,
należało zastosować szereg technik memoizacji tj. zapamiętywania raz wyliczonych
wartości, bez konieczności kalkulowania ich za każdym razem gdy symulowane
wyżarzanie wykonuję jedną ze swoich iteracji. Tak jest w przypadku kary, która
jest sumą pomniejszych wartości. Te pomniejsze wartości zmieniają się tylko w
przypadku zmiany kompozycji danego dnia. Taki mechanizm pozwolił znacząco
zredukować czas działania algorytmu.

\subsubsection{Efektywne klonowanie rozwiązań}
SA w każdej pętli przeprowadza przynajmniej jedno klonowanie rozwiązania.
Rozwiązania składają się z kilku pól zarówno tych o typie referencyjnym (z ang.
\emph{reference type}) jak i wartościowym (z ang. \emph{value type}). Kolekcje
zawarte w rozwiązaniu nie były klonowane, lecz miały kopiowaną między sobą
zawartość. Kopiowanie zawartości pozwoliło na znaczne przyśpieszenie działania
symulowanego wyżarzania oraz na ograniczenie niepotrzebnej alokacji pamięci.
