\chapter{Propozycja rozwiązania problemu harmonogramowania zabiegów}
Znając specyfikę problemu oraz metaheurystykę, która pomoże w jego rozwiązaniu,
można przejść do opisu proponowanego rozwiązania. W następnych sekcjach został
opisany projekt i implementacja systemu, odpowiedzialnego za harmonogramowanie zabiegów rehabilitacyjnych.

\section{Opis pojedynczej instancji problemu}
\section{Opis rozwiązania problemu}
\section{Lista ograniczeń twardych}
\section{Lista preferencji}
\section{Wykorzystane schematy chłodzenia}
\subsection{Chłodzenie liniowe}
\subsection{Chłodzenie kwadratowe}
\subsection{Cykliczne podgrzewanie}
\section{Dobór parametrów}

Jak zostało przedstawione w rozdziale \ref{chapter:sa-desc}, algorytm
symulowanego wyżarzania korzysta z listy parametrów, które mają decydujący wpływ
na specyfikę jego zachowania.

W tym rozdziale opisany został proces, mający na celu wyłonienie najlepszych
parametrów, z uwzględnieniem różnych instancji jak i przestrzeni rozwiązań.
