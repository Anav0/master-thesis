\chapter{Propozycja rozwiązania problemu harmonogramowania zabiegów}
\label{solution}
Znając specyfikę problemu oraz metaheurystykę, która pomoże w jego rozwiązaniu,
można przejść do opisu proponowanego rozwiązania. W następnych sekcjach został
opisany projekt i implementacja systemu, odpowiedzialnego za harmonogramowanie zabiegów rehabilitacyjnych.

\section{Implementacja poszczególnych elementów SA}
Schemat działania symulowanego wyżarzania z listingu \ref{alg-annealing} wymaga
zdefiniowania pewnych niezbędnych elementów takich jak: $Eval(sol)$,
$Change(sol)$,$Cool(sol)$, $initialSol$, $initialTemp$ etc. Kolejne sekcje tego rozdziału
szczegółowo opisują rzeczywistą implementację tychże elementów. 
\pagebreak
\subsection{Pojedyncza instancja problemu}
Problem harmonogramowania zabiegów rehabilitacyjnych został zdefiniowany w
rozdziale \ref{problem-definition} jako: $SP=\{\sRef,start,end\}$, gdzie \sRef\ jest
konkretnym zleceniem, złożonym z $n$ pomniejszych zleceń \sSubRef\ na konkretny zabieg. W oparciu o te informacje możemy przedstawić pojedynczą instancję problemu jako:
\begin{table}[h]
	\centering
	\begin{tabular}{ | m{0.3\textwidth} | m{0.7\textwidth} | }
	\hline
	$R_{id}=[z_1,z_2...z_n]$ & $n$ elementowa tablica zawierająca id zabiegów. \\
	\hline
	$R_{dur}=[d_1,d_2...d_n]$ & $n$ elementowa tablica zawierająca czas trwania zabiegów. \\
	\hline
	$P_{id}$ & identyfikator pacjenta dla którego wyznaczamy zabiegi. \\
	\hline
	$G$ & słownik zawierający w sobie informację o wymaganych przerwach
	między zabiegami. Kluczem słownika jest krotka $(z_x,z_y)$, gdzie $z_x$
	i $z_y$ to id zabiegów między którymi ma być zachowana przerwa a
	wartość jest to liczba określająca minimalną przerwę w minutach. \\
	\hline
	$start,\ end$ & daty określające ramy wyznaczania. \\
	\hline
	$Trms = [trm_1,trm_2...trm_j]$ & jest listą wszystkich terminów między
	$start$ i $end$ posortowaną według daty ich rozpoczęcia.
	Pojedynczy termin jest strukturą zawierającą informację o dacie
	swojego rozpoczęcia, miejscu wykonania zabiegu i jego pojemności, czasie trwania, oraz
	zabiegu jaki może być do niego przypisany.\\
	\hline
\end{tabular}
\caption{Pojedyncza instancja problemu rehabilitacyjnego}
\end{table}
\subsection{Reprezentacja rozwiązania}
Rozwiązanie zostało zdefiniowane w rozdziale \ref{problem} jako:
$\sSol=\{\sTermSet_1,\sTermSet_2...\sTermSet_n\}$, gdzie $\sTermSet_i$ jest
zbiorem terminów \sTerm\ w których ma się odbyć i'ty zabieg rehabilitacyjny.
\begin{table}[h]
	\begin{tabular}{ | m{0.2\textwidth} | m{0.8\textwidth} | }
		\hline
		$S_{trm}=[s_1,s_2...s_n]$ & $n$ elementowa tablica zawierająca tablice
		(tablica tablic) wypełnioną id terminów \\
		\hline
		$I_{trm}=[p_1,p_2...p_n]$ & $n$ elementowa tablica zawierająca pozycję i'tego terminu w liście $Term$.\\
		\hline
\end{tabular}
\caption{Pojedyncze rozwiązanie problemu rehabilitacyjnego}
\end{table}
Pozycja elementów $s$ w tablicy $S_{trm}$ odpowiada konkretnemu zabiegowi $z$ w tablicy
$R_{id}$. $s_i$ jest zbiorem terminów wyznaczonych dla zabiegu o id $z_i$.
\subsection{Ocena rozwiązania}
Wiedząc jak program reprezentuje instancję problemu oraz jego potencjalne
rozwiązanie, można określić funkcję odpowiedzialną za ocenę generowanych
rozwiązań.

Zgodnie z przedstawionym schematem z rysunku \ref{alg-annealing}, symulowane
wyżarzanie

Wartość rozwiązania jest opisana przez jedną z dwóch liczb. Jeśli rozwiązanie
jest dopuszczalne wartością rozwiązania jest wynik funkcji $V(sol)$. Jeśli
rozwiązanie jest niedopuszczalne jest to wynik funkcji $P(sol)$.

Mamy dane rozwiązanie $sol$, które chcemy sprawdzić pod kątem dopuszczalności.
Funkcja obliczająca karę wygląda następująco: $P(sol) = \sum_m^{i=0} P_i(sol)$, gdzie:
\begin{center}
	\begin{tabular}{ m{0.2\textwidth} | m{0.8\textwidth} }
		$m$ &  liczba pomniejszych funkcji kary $P_1,P_2...P_m$\\
		$P_1(sol)$ & zwraca ilość zbiorów terminów $s \in S_{trm}$, które zaczynają się w tym
		samym dniu i są dla tego samego typu zabiegu. \\
		$P_2(sol)$ & zwraca ilość zbiorów terminów $s \in S_{trm}$,
		które nie zachowują odpowiedniej przerwy określonej w słowniku
		$G$. \\
		$P_3(sol)$ & funkcja sprawdzająca spełnienie kryterium $C_3$ \\
		$P_4(sol)$ & funkcja sprawdzająca spełnienie kryterium $C_4$ \\
\end{tabular}
\end{center}
\begin{enumerate}
	\item{$C_1$ - Dwa zabiegi rehabilitacyjne tego samego typu nie mogą być
		zaplanowane na ten sam dzień.}
	\item{$C_2$ - Dwa zabiegi rehabilitacyjne nie mogą być zaplanowane w na ten sam czas.}
	\item{$C_3$ - Zabiegi określone w słowniku \sGap\ muszą zachować określony tam odstęp.}
	\item{$C_4$ - Zabiegi nie mogą zostać zapisane w terminach, które są zapełnione.}
	\item{$C_5$ - Terminy przypisane danemu zabiegowi muszą być po sobie następujące.}
\end{enumerate}
\subsection{Lista preferencji}
\subsection{Wykorzystane schematy chłodzenia}
