\newcolumntype{L}[1]{>{\hsize=#1\hsize\raggedright\arraybackslash}X}%
\newcolumntype{R}[1]{>{\hsize=#1\hsize\raggedleft\arraybackslash}X}%
\newcolumntype{C}[1]{>{\hsize=#1\hsize\centering\arraybackslash}X}%

\chapter{Propozycja rozwiązania problemu harmonogramowania zabiegów}
\label{solution}
Znając specyfikę problemu oraz metaheurystykę, która pomoże w jego rozwiązaniu,
można przejść do opisu proponowanego rozwiązania. W następnych sekcjach został
opisany projekt i implementacja systemu, odpowiedzialnego za harmonogramowanie zabiegów rehabilitacyjnych.

\section{Wykorzystane technologie}
Program został napisany w technologi .Net Core
5.0.\footnote{\url{https://docs.microsoft.com/en-us/dotnet/fundamentals/} -
dokumentacja .Net Core} Interfejs użytkownika został stworzyny za pomocją języka
Typescript i
framework'a svelte\footnote{\url{https://svelte.dev/docs} - dokumentacja
framwework'a svelte}. 

\section{Przypadek użycia}
Użytkownik za pomocą interfejsu wybiera zlecenie
do wyznaczenie. Po wyznaczeniu proponowane rozwiązanie jest wyświetlone.
Użytkownik może dostosować propozycję według własnego uznania. Zmiany
wprowadzone przez użytkownika są automatycznie sprawdzane przez system pod
kątem spełnienia ograniczeń. Po zatwierdzeniu propozycji, program raz jeszcze
sprawdza dopuszczalność rozwiązania i jeśli rozwiązanie wciąż jest dopuszczalne,
zostaje zapisane w bazie danych.
\newpage
\section{Implementacja poszczególnych elementów SA}
Schemat działania symulowanego wyżarzania z listingu \ref{alg-annealing} wymaga
zdefiniowania pewnych niezbędnych elementów takich jak: $Eval(sol)$,
$Change(sol)$, $Cool(sol)$, $initialSol$, $initialTemp$ etc. Kolejne sekcje tego rozdziału
szczegółowo opisują rzeczywistą implementację tychże elementów. 

\subsection{Reprezentacja pojedynczej instancji}
Bazując na opisie z rozdziału \ref{problem-desc}, możemy przedstawić instancję
problemu jako:
\begin{table}[h]
	\centering
	\begin{tabular}{ | m{0.3\textwidth} | m{0.7\textwidth} | }
	\hline
	$R_{id}=[z_1,z_2...z_n]$ & $n$ elementowa tablica zawierająca id zabiegów. \\
	\hline
	$R_{dur}=[d_1,d_2...d_n]$ & $n$ elementowa tablica zawierająca czas trwania zabiegów. \\
	\hline
	$P_{id}$ & identyfikator pacjenta dla którego wyznaczamy zabiegi. \\
	\hline
	$G$ & słownik zawierający w sobie informację o wymaganych przerwach
	między zabiegami. Kluczem słownika jest krotka $(z_x,z_y)$, gdzie $z_x$
	i $z_y$ to id zabiegów między którymi ma być zachowana przerwa a
	wartość jest to liczba określająca minimalną przerwę w minutach. \\
	\hline
	$start,\ end$ & daty określające ramy wyznaczania. \\
	\hline
	$Trms = [trm_1,trm_2...trm_j]$ & jest listą wszystkich terminów między
	$start$ i $end$ posortowaną według daty ich rozpoczęcia.
	Pojedynczy termin jest strukturą zawierającą informację o dacie
	swojego rozpoczęcia, miejscu wykonania zabiegu i jego pojemności, czasie trwania, oraz
	zabiegu jaki może być do niego przypisany.\\
	\hline
\end{tabular}
\caption{Pojedyncza instancja problemu rehabilitacyjnego}
\end{table}
\subsection{Reprezentacja rozwiązania}
Symulowane wyżarzanie musi przechowywać informację o trzech rozwiązaniach:
$sol$, $bestSol$ oraz $beforeChange$. W przypadku złego ruchu - takiego, które
nie zostało uznane za warte przeprowadzenia - program musi głęboko skopiować
jedno rozwiązanie - to przed zmianą - do drugiego - tego po zmianie
(linijka \ref{annealing-copy} algorytmu). Aby ten
proces był jak najefektywniejszy, warto zadbać o to, aby typy pól
naszego rozwiązania było łatwo klonowalne.

Rozwiązanie zostało opisane w sekcji \ref{problem-desc} jako de facto zbiór par: \emph{id
zlecenia} - \emph{id terminu} przyczym jedno zlecenie może mieć przypisany
więcej niż jeden termin. 

Mając ten opis w pamięci, rozwiązanie możemy przedstawić jako:
$\sSol=\{\sTermSet_1,\sTermSet_2...\sTermSet_n\}$, gdzie $\sTermSet_i$ jest
zbiorem terminów \sTerm\ w których ma się odbyć i'ty zabieg rehabilitacyjny.
Pozycja elementów $s$ w tablicy $S_{trm}$ odpowiada konkretnemu zabiegowi $z$ w
tablicy $R_{id}$. $s_i$ jest zbiorem terminów wyznaczonych dla zabiegu o id
$z_i$.
\begin{table}[h]
	\begin{tabular}{ | m{0.2\textwidth} | m{0.8\textwidth} | }
		\hline
		$S_{trm}=[s_1,s_2...s_n]$ & $n$ elementowa tablica zawierająca tablice
		(tablica tablic) wypełnioną identyfikatorami terminów \\
		\hline
		$I_{trm}=[p_1,p_2...p_n]$ & $n$ elementowa tablica zawierająca pozycję i'tego terminu w liście $Term$.\\
		\hline
\end{tabular}
\caption{Pojedyncze rozwiązanie problemu rehabilitacyjnego}
\end{table}
\newpage
\subsection{Ocena rozwiązania}
Wiedząc jak program reprezentuje instancję problemu oraz jego potencjalne
rozwiązanie, można określić funkcje odpowiedzialne za ocenę rozwiązań
(\nameref{preferences} \ref{preferences} i \nameref{constraints} \ref{constraints}).

Symulowane wyżarzanie do porównywania jakości rezultatów używa liczby (linijka
\ref{annealing-comp} algorytmu). Jeśli rozwiązanie jest dopuszczalne, wartością
rozwiązania jest wynik funkcji $V(sol)$. Jeśli rozwiązanie jest niedopuszczalne
jest to wynik funkcji $K(sol)$.
\begin{equation}
	K(sol) = \sum_{n=2}^{i=1} K_i(sol)
\end{equation}

\begin{table}[h]
\begin{tabularx}{\textwidth}{ | C{0.4} | L{2.3} | C{0.3} | }		
	\hline
		$K_1(sol)$ & Zwraca ilość zbiorów terminów $s \in S_{trm}$, które zaczynają się w tym
		samym dniu i są dla tego samego typu zabiegu. & $C_1$ \\
		\hline
		$K_2(sol)$ & Zwraca ilość zbiorów terminów $s \in S_{trm}$,
		które nie zachowują odpowiedniej przerwy określonej w słowniku
		$G$. & $C_3$ \\
		\hline
\end{tabularx}
\end{table}

\begin{equation}
	V(sol) = \sum_{n=5}^{i=1} W(v_i) * v_i
\end{equation}

$W(v_i)$  Zwraca wagę dla danej pomniejsze wartości oceny.
\begin{table}[h]
	\begin{tabularx}{\textwidth}{ | C{0.4} | L{2.6} | C{0.3} | C{0.7} | }
		\hline
		\bfseries Symbol & \bfseries Opis & \bfseries Waga & \bfseries
		Preferencja \\
		\hline
		$v_1$ & Suma odległości, między zajętymi dniami a \emph{start}.&
		1 & $P_1$\\
		\hline
		$v_2$ & Suma ilość wyznaczonych
		wizyt, normalizuje wynik i odejmuje go od 1 aby uzyskać wartość
		poprawną z punktu widzenia problemu minimalizacji. Im więcej
		wizyt jednego dnia, tym lepiej. & 1 &
		$P_2$\\
		\hline
		$v_3$ & Sumuje dla każdego zajętego dnia różnicę między momentem wyjścia z
		kliniki (koniec najpóźniejszego terminu w danym dniu) a momentem
		przyjścia (start najwcześniejszego terminu w danym
		dniu). & 1.5 & $P_3$ i $P_4$ \\
		\hline
		$v_4$ & Suma różnicy między
		najwcześniejszym możliwym terminem w danym dniu a tym faktycznie
		najwcześniejszym. & 0.25 & $P_5$\\
		\hline
		$v_5$ & Znormalizowana ilość zajętych dni. Im mniej tym lepiej. & 1 & $P_6$\\
		\hline
\end{tabularx}
\end{table}


\subsubsection{Niewymienione ograniczenia}
Ograniczenia $C_2$ i $C_5$ nie mogą zostać złamane ze względu na specyfikę
ruchu, która to uniemożliwia. Ruch został szczegółowo opisany w kolejnej sekcji.
Ograniczenie $C_4$ jest rozwiązane na początku działania programu, ponieważ
lista \sTermsList\ jest zapełniona tylko takimi terminami, których sala ma
przynajmniej jedno miejsce wolne.

\subsubsection{Normalizacja} 
Każda pomniejsza wartość oceny jest normalizowana ostateczną sumą tak, aby mieściła się w
zakresie od 0 do 1. Liczba 1 oznacza najgorszy możliwy wynik dla
danego dnia a 0 najlepszy. Bez takiej normalizacji niemożliwe byłoby użycie wag.

\subsubsection{Średnia} 
Wyniki pomniejszych wartości oceny za wyjątkiem $v_4$ są dodatkowo na końcu dzielone przez ilość
wykorzystanych dni. Jeśli do zaplanowania zlecenia użyto 10 terminów - 6 zostało
przewidzianych na poniedziałek a 4 na wtorek - to ilość wykorzystanych dni jest
równa 2.

\begin{algorithm}
	\caption{Funkcja oceny $V(sol)$}\label{fn-obj}
\begin{algorithmic}[1]
\Procedure{V}{$sol$}
\State $v1,v2,v3,v4,v5 \gets 0$\Comment{Pomniejsze wartości oceny}
\State $start \gets GetStart()$\Comment{Daty definiujące zakres wyznaczania}
\State $end \gets GetEnd()$
\State $numberOfDays \gets 0$
\State $max \gets MaxBasedOnReferral()$

\For{ $day\ in\ UsedDays()$ }\Comment{Dni wykorzystane przez $s \in S$}
\State $numberOfDays \gets numberOfDays+1$
\State $numberOfVisits \gets GetVisitsFor(day)$

\State $firstTerm \gets First(day)$\Comment{Najwcześniejszy termin.}
\State $lastTerm \gets Last(day)$\Comment{Najpóźniejszy termin.}

\State $v1 \gets v1 + Norm(day-start,start,end)$
\State $v2 \gets v2 + (1 - Norm(numberOfVisits,0,max))$

\If{$numberOfVisits > 1$}
\State $v3 \gets v3 + Norm(lastTerm.StartDate-firstTerm.EndDate,start,end)$
\EndIf

\State $v4 \gets Norm(firstTerm.StartDate,start,end)$

\EndFor
\State $v1 \gets v1*W(v1)$
\State $v2 \gets v2*W(v2)$
\State $v3 \gets v3*W(v3)$
\State $v4 \gets v4*W(v4)$
\State $v5 \gets Norm(numberOfDays,0,TotalDays())*W(v5)$

\State \textbf{return} $((v1+v2+v3+v4)/numberOfDays)+v5$
\EndProcedure
\end{algorithmic}
\end{algorithm}

\begin{algorithm}
	\caption{Funkcja normalizująca}\label{fn-normalize}
\begin{algorithmic}[1]
\Function{$Norm$}{$value,min,max$}
\State \textbf{return} $(value-min)/(max-min)$
\EndFunction
\end{algorithmic}
\end{algorithm}
\newpage
\subsection{Zmiana rozwiązania}
W każdej iteracji pętli $while$ algorytm zmienia bieżące rozwiązanie. Funkcja
$Change(sol)$ zmienia jeden losowy zbiór terminów ze struktury $S_{trm}$.
Po wybraniu losowego $s \in S_{trm}$ następuje poszukiwanie nowych terminów $r$,
które mogłyby go wypełnić. Losowany jest nowy termin z listy $Trms$, on będzie
potencjalnie pierwszym terminem zbioru $r$. Jeśli termin ten jest bezkonfliktowy
i jego czas trwania jest większy lub równy długości trwania zabiegu, wtedy ruch
jest zakończony. Jeśli termin jest bezkonfliktowy ale jego długość nie pozwala
na odbycie terminu, należy znaleźć kolejne terminy, które umożliwią odbycie
zabiegu. Jeśli nie można znaleźć sąsiednich terminów lub termin jest
konfliktowy, wtedy początkowy termin $r[0]$ jest zamieniany na swojego
najbliższego niesprawdzonego sąsiada i cały proces sprawdzenia i ewentualnego
poszukiwania sąsiadów jest powtarzany.

\begin{algorithm}
	\caption{Funkcja zmieniająca rozwiązanie}\label{fn-change}
\begin{algorithmic}[1]
\Procedure{$Change$}{$sol$}
\State $i \gets Rand(0,sol.S_{trm}.length)$\Comment{Wybór losowego indeksu}
\State $s \gets AtIndex(i,sol.S_{trm})$\Comment{Terminy $s$ do zmiany}
\State $t \gets AtIndex(i,sol.R_{dur})$\Comment{Czas trwania zabiegu}
\State $r \gets EmptyArr()$
\State $r[0] \gets Rand(sol.Terms)$\Comment{Wybór losowego terminu początkowego}
\State $r \gets FindNearbyTerms(r[0], t_{dur})$
\While{$IsInConflict(r,sol.S_{trm})$}
\State $r[0] \gets Neighbour(r[0],sol.Terms)$\Comment{Najbliższy niesprawdzony sąsiad}
\State $r \gets FindNearbyTerms(r[0], t_{dur})$
\EndWhile
\State $s \gets r$
\EndProcedure
\end{algorithmic}
\end{algorithm}
Wykorzystanie funkcji $IsInConflict$ i $FindNearbyTerms$ nie dopuszcza do sytuacji łamiącej
kryteria $C_2$ i $C_5$ (\ref{constraints}).

\subsection{Kryterium stopu}
Dwa kryteria stopu zostały zastosowane. Kryterium maksymalnej ilości kroków oraz
minimalnej poprawy \ref{sec:stop-criteria} zostały zastosowane razem.

\subsection{Schemat chłodzenia}
\subsection{Konstrukcja rozwiązania początkowego}
