\chapter{Podsumowanie}
Celem pracy było opracowanie rozwiązania problemu harmonogramowania zabiegów
rehabilitacyjnych. Praca rozpoczyna się od opisu samego problemu, jego
specyfiki, niuansów i szczegółów. Następnie przeprowadzony został przegląd
literatury, gdzie najważniejsze techniki możliwe do zastosowania zostały
opisane. Algorytm symulowanego wyżarzania, który posłużył jako główne narzędzie
do rozwiązywania problemu został przedstawiony a razem z nim szczegóły
implementacyjne programu rozwiązującego przedstawiony problem.

Po zaproponowaniu rozwiązania, przeprowadzone zostało badanie jego efektywności.
Wybrano najczęściej występujące instancje różnych wielkości i sprawdzono jakie
średnie wartości rozwiązania są uzyskiwane. Wartości uzyskane dla każdej
instancji były zadowalające a czas działania dostatecznie niski.

Opracowane rozwiązanie może zostać użyte jako skuteczne narzędzie pomocnicze
wspomagające pracowników w procesie harmonogramowania zabiegów
rehabilitacyjnych.
