\chapter{Cel i zakres pracy}
Celem niniejszej pracy jest rozwiązanie problemu wyznaczania szeregu terminów rehabilitacji za pomocą algorytmu symulowanego wyżarzania. Oprócz samego rozwiązania problemu, szczegółowo zostanie opisany wpływ następujących czynników na szybkość działania algorytmu oraz na jakość uzyskiwanych rozwiązań:

\begin{itemize}
\item Temperatura początkowa
\item Funkcja wychładzania
\item Typ funkcji przejścia
\item Wielkość instancji (ilość wizyt do wyznaczenia)
\item Wielkość przestrzeni rozwiązań
\item Ograniczenia twarde
\item Ograniczenia miękkie
\end{itemize}
Po zbadaniu wpływu powyższych czynników na czas działania oraz jakość uzyskanych rozwiązań, różne połączenia parametrów - za wyjątkiem ograniczeń - również zostaną przebadane a wyniki szczegółowo opisane. 

Szczególna uwaga zostanie poświęcona zbadaniu relacji między wielkością instancji problemu a parametrami, dokładniej: czy optymalnie dobrane parametry dla pewnej wielkości instancji dadzą równie dobre rezultaty dla instancji o innej wielkości, oraz innej kompozycji (zawierającej inne typy wizyt).

\chapter{Wstęp}
Optymalizacja procesów w placówkach medycznych jest 
Rozwiązanie problemu wyznaczania terminów rehabilitacji polega na znalezieniu takich bloków
czasowych (blok trwa 20min), w których mogą zostać umieszczone terminy wykonania zabiegów. Każdy
pacjent ma listę zabiegów do wykonania (np. 2x bieganie, 3x masaż). 

Dane zebrane przez placówkę rehabilitacyjną pokazują, że średnia ilość wyznaczanych zabiegów dla jednego pacjenta to 80. Zabiegi mogą zostać wykonane tylko w pewnej sali zabiegowej, która znajduje się w danym bloku czasowych i ma swoją ograniczoną pojemność.

Pewne zabiegi wymagają konkretnego odstępu czasowego między sobą np. zabieg
\emph{laseroterapii} nie może być wykonywany w przeciągu 1h od momentu zakończenia
wykonania zabiegu \emph{krioterapii}. Taka reguła nie koniecznie może występować w obie
strony.

Terminy preferowane przez pacjenta, oraz godziny otwarcia danej placówki też
muszą być uwzględnione w procesie wyznaczania terminów odbycia zabiegów - tak samo
jak święta czy sytuacje nagłe takie jak zepsucie sprzętu niezbędnego do
przeprowadzenia danej procedury.

Dostępność terminów jest bardzo płynna i często zdarzają się sytuacje w których
dany zabieg może być wyznaczony na drugi dzień od momentu otrzymania zgłoszenia, a już kolejny zabieg na liście
zabiegowej za 4 miesiące. Dlatego też przedział czasowy brany pod uwagę musi być odpowiednio duży (od 6
miesięcy do roku).
\vskip 2em \parindent 0em Jak widać po tym, krótkim i niepełnym opisie problem nie jest trywialny. Duża przestrzeń potencjalnych rozwiązań, typów zabiegów oraz ograniczeń - o których więcej w rozdziale \ref{constraints} - sprawia, że problem jest NP trudny.

\chapter{Modelowanie problemu wyznaczania terminów rehabilitacji}
Ten rozdział przedstawia pewną grupę problemów. Są to problemy,
które można nazwać problemami harmonogramowania (z ang. scheduling problems).
Do tej grupy można zaliczyć takie problemy jak: \emph{Job Shop Problem} czy
\emph{Nurse Scheduling Problem }. Wszystkie problemy, które możemy zaklasyfikować
jako problemy harmonogramowania, starają się jak najlepiej przypisać pewne
zadania (np. rehabilitację) do pewnych terminów, uwzględniając przy tym ograniczone zasoby (np. ilość
sal, czy maszyn).

Opisanie zróżnicowania problemów istniejących w tej grupie, jest ważne w celu pełnego
zrozumienia, możliwych podejść do rozwiązania problemu wyznaczania
terminów rehabilitacji i wybraniu najlepszego.

\section{Opis istniejących problemów harmonogramowania}
\subsection{Job Shop Problem}
\subsection{Nurse Scheduling Problem \cite{nurseScheduling}}
\subsection{College Admission Problem (Matching) \cite{matchingUnderPref}}
Ten problem stara się przypisać rezydentów z listy $R={r_1,r_2,r_3...r_k}$ do szpitali z listy
$H={h_1,h_2...h_n}$ w taki sposób aby pojemność danego szpitala $c_j$ nie
została przekroczona, oraz aby szpital do którego zostanie przypisany rezydent był przez niego akceptowalny i vice versa. 

Niestety modelowanie problemu za pomocą matchingu wymaga przypisania do
siebie elementów dwóch lub więcej zbiorów. Jest to kłopotliwe w problemie wyznaczania
terminów rehabilitacji. Zbiór zabiegów należałoby połączyć ze zbiorem
terminów, tylko że zbiór terminów jest bardzo duży, dodatkowo nie jesteśmy w
stanie określić wartości takiego połączenia między zabiegiem a terminem w
izolacji od innych dat wykonania innych zabiegów, tych istniejących jak i tych
obecnie wyznaczanych.

\chapter{Przedstawienie problemu jako ogólnego problemu optymalizacji}
\section{Opis instancji problemu}
\section{Charakterystyka przestrzeni rozwiązań}
\section{Funkcja oceny rozwiązania}
\section{Lista ograniczeń twardych \label{constraints}}
\subsection{Ograniczenie ilości wizyt w jednym dniu}
\subsection{Wizyty nie mogą na siebie nachodzić}
\subsection{itd itp (nie wiem czy warto je tak wypisywać jako sekcje}
\section{Lista ograniczeń miękkich}
\subsection{Preferuj czas podany przez pacjenta}
\section{Złożoność pamięciowa przyjętego modelu (Może być za dużo roboty)}
\section{Złożoność obliczeniowa przyjętego modelu (Może być za dużo roboty)}

\chapter{Zastosowane algorytmy metaheurystyczne}
\section{Symulowane wyżarzanie}
\subsection{Schemat działania symulowanego wyżarzania}
\subsection{Ruchy zmieniające rozwiązanie}
\subsection{Kryteria stopu}
\subsection{Temperatura początkowa}
\subsection{Funkcje wychładzania}
\section{Symulowane wyżarzanie wzbogacone o tabelę Tabu}
\subsection{Funkcja utrzymania tabu}
%\section{Algorytm genetyczny (optional)}
%\subsection{Populacja początkowa}
%\subsection{Operator krzyżowania}
%\subsection{Operator mutacji}

\addtocontents{toc}{\protect\newpage}

\chapter{Analiza algorytmu symulowanego wyżarzania}
\section{Opis stałych elementów analizy}
\subsection{Opis przyjętych instancji problemu}
\subsection{Opis przyjętej przestrzeni rozwiązań}
\section{Przeprowadzone badania}
\subsection{Wpływ temperatury początkowej na jakość rozwiązania}
\subsection{Wpływ funkcji wychładzania na jakość rozwiązania}
\subsection{Wpływ funkcji ruchu na jakość rozwiązania}
\subsection{Wpływ wielkości instancji na jakość rozwiązania}
\subsection{Jakość rozwiązania uzyskana po wymieszaniu parametrów}
\subsection{Jakość rozwiązania uzyskana po połączeniu najlepszych parametrów}

\chapter{Podsumowanie}
