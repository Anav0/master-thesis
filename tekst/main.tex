\chapter{Cel i zakres pracy}
Celem niniejszej pracy jest rozwiązanie problemu wyznaczania szeregu terminów rehabilitacji za pomocą algorytmu symulowanego wyżarzania. Oprócz samego rozwiązania problemu, szczegółowo zostanie opisany wpływ następujących czynników na szybkość działania algorytmu oraz na jakość uzyskiwanych rozwiązań:

\begin{itemize}
\item Temperatura początkowa
\item Funkcja wychładzania
\item Typ funkcji przejścia
\item Wielkość instancji (ilość wizyt do wyznaczenia)
\item Wielkość przestrzeni rozwiązań
\item Ograniczenia twarde
\item Ograniczenia miękkie
\end{itemize}
Po zbadaniu wpływu powyższych czynników na czas działania oraz jakość uzyskanych rozwiązań, różne połączenia parametrów - za wyjątkiem ograniczeń - również zostaną przebadane a wyniki szczegółowo opisane.

Szczególna uwaga zostanie poświęcona zbadaniu relacji między wielkością instancji problemu a parametrami, dokładniej: czy optymalnie dobrane parametry dla pewnej wielkości instancji dadzą równie dobre rezultaty dla instancji o innej wielkości, oraz innej kompozycji (zawierającej inne typy wizyt).

\chapter{Modelowanie problemu wyznaczania terminów rehabilitacji}
Ten rozdział przedstawia pewną grupę problemów. Są to problemy,
które można nazwać problemami harmonogramowania (z ang. scheduling problems).
Do tej grupy można zaliczyć takie problemy jak: \emph{Job Shop Problem} czy
\emph{Nurse Scheduling Problem}. Wszystkie problemy, które możemy zaklasyfikować
jako problemy harmonogramowania, starają się jak najlepiej przypisać pewne
zadania (np. odbycie zabiegu rehabilitacyjnego) do pewnych terminów, uwzględniając przy tym ograniczone zasoby (np. ilość
sal, czy maszyn).

Opisanie zróżnicowania problemów istniejących w tej grupie, jest ważne w celu pełnego
zrozumienia, możliwych podejść do rozwiązania problemu wyznaczania
terminów rehabilitacji i wybraniu najlepszego.

\section{Opis istniejących problemów harmonogramowania}
\subsection{Job Shop Problem}
\subsection{Problem harmonogramowania dyżurów pielęgniarek}
Problem harmonogramowania dyżurów pielęgniarek\cite{nurseScheduling} (\emph{z ang. NSP}) jest problemem często spotykanym nie tylko w placówkach medycznych, ale
wszędzie tam gdzie istnieje potrzeba ustalenia grafiku dla wielu pracowników.

Ustalenie takiego harmonogramu ręcznie jest czasochłonne oraz nieefektywne,
szczególnie jeśli chcemy uwzględnić cały szereg ograniczeń, nałożonych przez
pracodawcę, umowy czy podpisane zobowiązania.

W swojej najogólniejszej postaci NSP stara się przypisać pracowników do
odpowiednich zmian w ciągu dnia tak, aby zmaksymalizować zadowolenie pracowników
oraz zminimalizować koszty pracodawcy przy jednoczesnym spełnieniu całej listy
wymagań.

Problem NSP jest bardzo zbliżonym problemem, do problemu który chcemy rozwiązać
w tej pracy. Jednak przestrzeń z którą mamy do czynienia w większości wariantów problemu NSP jest o wiele mniejsza - od 7 do 30 dni, trzy zmiany w każdym dniu - niż w problemie wyznaczania terminów rehabilitacji - od 60 dni do 365 dni każdy podzielony na 20min bloki.

Dodatkowo w procesie wyznaczania terminów rehabilitacji musimy uwzględnić już wsześniej wyznaczone terminy, tak aby spełnić ograniczenie narzucającące nam zachowanie odstepów czasowych między pewnymi typami zabiegów.

Niestety modelowanie problemu za pomocą matchingu wymaga przypisania do
siebie elementów dwóch lub więcej zbiorów. Jest to kłopotliwe w problemie wyznaczania
terminów rehabilitacji. Zbiór zabiegów należałoby połączyć ze zbiorem
terminów, tylko że zbiór terminów jest bardzo duży. Dodatkowo nie jesteśmy w
stanie określić wartości połączenia między zabiegiem a terminem w
izolacji od innych dat wykonania zabiegów, tych istniejących jak i tych
obecnie wyznaczanych.

\chapter{Przedstawienie problemu jako ogólnego problemu optymalizacji}
\section{Opis instancji problemu}
\section{Charakterystyka przestrzeni rozwiązań}
\section{Funkcja oceny rozwiązania}
\section{Lista ograniczeń twardych \label{constraints}}
\subsection{Ograniczenie ilości wizyt w jednym dniu}
\subsection{Wizyty nie mogą na siebie nachodzić}
\subsection{itd itp (nie wiem czy warto je tak wypisywać jako sekcje}
\section{Lista ograniczeń miękkich}
\subsection{Preferuj czas podany przez pacjenta}
\section{Złożoność pamięciowa przyjętego modelu (Może być za dużo roboty)}
\section{Złożoność obliczeniowa przyjętego modelu (Może być za dużo roboty)}

\chapter{Opis elementów symulowanego wyżarzania}

\section{Schemat działania symulowanego wyżarzania}

\section{Metody zmiany rozwiązania}
\subsection{Zmiana całkowicie losowa}
\subsection{Zmiana stopniowa}
\subsection{Zmiana inteligentna}

\section{Kryteria stopu}
\subsection{Maksymalna ilość kroków}
\subsection{Maksymalna ilość kroków bez znalezienia lepszego rozwiązania}
\subsection{Aż do odpowiedniego spadku temperatury}

\section{Funkcje wychładzania}
\subsection{Chłodzenie liniowe}
\subsection{Chłodzenie kwadratowe}
\subsection{Chłodzenie krokowe}
\subsection{Podgrzewanie}

\addtocontents{toc}{\protect\newpage}

\chapter{Analiza algorytmu symulowanego wyżarzania}
\section{Opis stałych elementów analizy}
\subsection{Opis przyjętych instancji problemu}
\subsection{Opis przyjętej przestrzeni rozwiązań}
\section{Przeprowadzone badania}
\subsection{Wpływ temperatury początkowej na jakość rozwiązania}
\subsection{Wpływ funkcji wychładzania na jakość rozwiązania}
\subsection{Wpływ funkcji ruchu na jakość rozwiązania}
\subsection{Wpływ wielkości instancji na jakość rozwiązania}
\subsection{Jakość rozwiązania uzyskana po wymieszaniu parametrów}
\subsection{Jakość rozwiązania uzyskana po połączeniu najlepszych parametrów}

\chapter{Podsumowanie}
