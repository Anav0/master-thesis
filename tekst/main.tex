
% Cel i zakres pracy
\chapter{Cel i zakres pracy} Celem niniejszej pracy jest rozwiązanie problemu
wyznaczania szeregu terminów rehabilitacji za pomocą algorytmu symulowanego
wyżarzania. Oprócz samego rozwiązania problemu, szczegółowo zostanie opisany
wpływ następujących czynników na szybkość działania algorytmu oraz na jakość
uzyskiwanych rozwiązań:

\begin{itemize} \item Temperatura początkowa \item Funkcja wychładzania \item
	Typ funkcji przejścia \item Wielkość instancji (ilość wizyt do
	wyznaczenia) \item Wielkość przestrzeni rozwiązań \item Ograniczenia
	twarde \item Ograniczenia miękkie \end{itemize} Po zbadaniu wpływu
	powyższych czynników na czas działania oraz jakość uzyskanych rozwiązań,
	różne połączenia parametrów - za wyjątkiem ograniczeń - również zostaną
	przebadane a wyniki szczegółowo opisane.

Szczególna uwaga zostanie poświęcona zbadaniu relacji między wielkością
instancji problemu a parametrami, dokładniej: czy optymalnie dobrane parametry
dla pewnej wielkości instancji dadzą równie dobre rezultaty dla instancji o
innej wielkości, oraz innej kompozycji (zawierającej inne typy wizyt).


\chapter{Przedstawienie problemu harmonogramowania zabiegów rehabilitacyjnych}
Ten rozdział ma na celu przybliżenie specyfiki problemu harmonogramowania zabiegów
rehabilitacyjnych, jaki występuje w placówkach współpracujących z firmą Medinet.

\chapter{Przegląd literatury}
W tym rozdziale zostały opisane wnioski, wyciągnięte z przeglądu dotychczasowych
prac badawczych traktujących o problemach harmonogramowania.

\chapter{Opis algorytmu symulowanego wyżarzania} \label{chapter:sa-desc}
Symulowane wyżarzanie \cite{metaheuristic-handbook} \emph{(z ang. simulated annealing)} (w skrócie: SA) jest metaheurystyką stosowaną, przy rozwiązywaniu szerokiego wachlarza problemów \emph{NP trudnych}. W tym rozdziale opisane zostaną kluczowe elementy składowe SA takie jak: funkcja celu, ograniczenia, schemat chłodzenia itd.

\section{Przedstawienie głównej pętli algorytmu wyżarzania}
\section{Sposoby zmiany rozwiązania} 
\section{Specyfika funkcji oceny rozwiązania}
\section{Specyfika funkcji kary za niedopuszczalność}
\section{Wpływ temperatury na akceptację rozwiązania}
\section{Schematy chłodzenia}

\chapter{Propozycja rozwiązania problemu harmonogramowania zabiegów}
Znając specyfikę problemu oraz metaheurystykę, która pomoże w jego rozwiązaniu,
można przejść do opisu proponowanego rozwiązania. W następnych sekcjach został 
opisany projekt i implementacja systemu, odpowiedzialnego za harmonogramowanie zabiegów rehabilitacyjnych.

\section{Opis pojedynczej instancji problemu}
\section{Opis rozwiązania problemu}
\section{Lista ograniczeń twardych}
\section{Lista preferencji}
\section{Wykorzystane schematy chłodzenia}
\subsection{Chłodzenie liniowe}
\subsection{Chłodzenie kwadratowe}
\subsection{Cykliczne podgrzewanie}
\section{Dobór parametrów}

Jak zostało przedstawione w rozdziale \ref{chapter:sa-desc}, algorytm
symulowanego wyżarzania korzysta z listy parametrów, które mają decydujący wpływ
na specyfikę jego zachowania. 

W tym rozdziale opisany został proces, mający na celu wyłonienie najlepszych
parametrów, z uwzględnieniem różnych instancji jak i przestrzeni rozwiązań.


\addtocontents{toc}{\protect\newpage}

\chapter{Badania efektywności zaproponowanego rozwiązania}
Rozdział ten opisuje badania mające na celu sprawdzenie efektywności
zaproponowanego rozwiązania. Wpływ różnych elementów algorytmu na czas działania programu, jak i jakość
uzyskiwanych rozwiązań został zbadany.

\section{Założenia metodologiczne}
\section{Opis czynników mogących wpłynąć na wyniki badań}

\section{Przedstawienie zużycia pamięci przez zaproponowany program}
INFO: TUTAJ BĘDZIE OPIS TEGO ILE PROCESOR SPĘDZA CZASU NA POSZCZEGÓLNYCH KROKACH. ILE
NA OGRANICZENIU X ILE NA Y A ILE NA Z. ILE NA OCENIĘ ROZWIĄZANIA; ILE NA
KONTAKT Z DB

\section{Przedstawienie czasu działania zaproponowanego programu} \label{section:memory-usage}
Rozdział \ref{section:memory-usage} przedstawia średnie zużycie pamięci
komputera odpowiedzialnego za proces harmonogramowania.

\section{Badanie wpływu temperatury początkowej na jakość rozwiązania}
\section{Badanie wpływu funkcji wychładzania na jakość rozwiązania}
\section{Badanie wpływu funkcji ruchu na jakość rozwiązania}
\section{Badanie wpływu ilości zabiegów w instancji problemu na jakość rozwiązania}

\chapter{Podsumowanie}
